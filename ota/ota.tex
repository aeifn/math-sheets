\documentclass[12pt]{article}
\usepackage[utf8]{inputenc}
\usepackage[russian]{babel}
\usepackage{amsmath}
\usepackage{setspace}
\usepackage{indentfirst}


\begin{document}
\section*{Основная теорема арифметики}

Ниже представлены задачи к лекции «Основная теорема арифметики» к курсу «100 уроков математики» Алексея Владимировича Саватеева.
Задачи разделены на 2 вида: типовые(на "прямое" применение основной теоремы арифметики) и нетиповые (решение этих задач потребует большего количества времени и «математической смекалки»).

\subsection*{Типовые задачи}

Задача 1. Не вычисляя произведения $2013\cdot15\cdot77$, выясните, делится ли оно на 2, 3, 9, 35, 55, 80, 6039.

Задача 2. Число $A$ делится на 3 и 4. Следует ли отсюда, что $A$ делится на $3\cdot4 = 12$?

Задача 3. Число $A$ делится на 4 и 6. Следует ли отсюда, что $A$ делится на $4\cdot6 = 24$?

Задача 4. Число $3A$ делится на 7. Следует ли отсюда, что $A$ делится на 7?

Задача 5. Докажите, что произведение трёх последовательных натуральных чисел делится на 6.

Задача 6. Докажите, что произведение пяти последовательных натуральных чисел делится на 120.

\subsection*{Нетиповые задачи}

Задача 1. Допишите к числу $523$ три цифры так, чтобы полученное шестизначное число делилось на 7, 8 и 9. Сколько всего таких чисел существует?

Задача 2. На сколько нулей оканчивается число $100!$?

Задача 3. В доме на всех этажах во всех подъездах равное количество квартир (больше одной). Также во всех подъездах поровну этажей. При этом количество этажей больше количества квартир на этаже, но меньше, чем ко- личество подъездов. Сколько в доме этажей, если всего квартир 715?

Задача 4. Укажите пять целых положительных чисел, сумма которых равна 20, а произведение 420.

Задача 5. В конце четверти Вовочка выписал подряд встрочку свои текущие отметки по пению и поставил между некоторыми из них знак умножения. Произведение получившихся чисел оказалось равным 2007. Какая отметка выходит у Вовочки
в четверти по пению? Колов учительница пения не ставит.

Задача 6. Есть четыре карточки с цифрами: 2, 0, 1, 6. Для каждого из чисел от 1 до 9 можно из этих карточек составить четырёхзначное число, которое кратно выбранному однозначному. А в каком году такое будет в следующий раз?

Задача 7. Охотник рассказал приятелю, что видел в лесу волка с метровым хвостом. Тот рассказал другому приятелю, что в лесу видели волка с двухметровым хвостом. Передавая новость дальше, простые люди увеличивали длину хвоста вдвое, а творческие - втрое. В результате по телевизору сообщили о волке с хвостом длиной 864 метра. Сколько простых и сколько творческих людей отрастили волку хвост?

Задача 9. Число умножили на сумму его цифр и получили 2008. Найдите это число.

Задача 10. Юра записал четырёхзначное число. Лёня прибавил к первой цифре этого числа 1, ко второй 2, к третьей 3 и к четвёртой 4, а потом перемножил полученные суммы. У Лёни получилось 234. Какое число могло быть записано Юрой?

Задача 11. Коробка с сахаром имеет форму прямоугольного параллелепипеда. В ней находится 280 кусочков сахара, каждый из которых кубик размером $1\times1\times1$ см. Найдите площадь полной поверхности коробки, если известно, что длина каждой из её сторон меньше 10 см.

Задача 12. На клетчатой бумаге нарисовали большой квадрат. Его разрезали на несколько одинаковых средних квадратов. Один из средних квадратов разрезали на несколько одинаковых маленьких квадратов. Стороны всех квадратов проходят по линиям сетки. Найдите длины сторон большого, среднего и маленького квадратов, если сумма их площадей равна 154.

Задача 13. Найти все натуральные числа n от 1 до 100 такие, что если перемножить все делители числа n (включая 1 и n), получим число $n^3$.

Задача 14. Используя в качестве чисел любое количество монет достоинством 1, 2, 5 и 10 рублей, а также (бесплатные) скобки и знаки четырёх арифметических действий, составьте выражение со значением 2009, потратив как можно меньше денег.

Задача 15. Произведение последовательных чисел от 1 до n называется n факториал и обозначается $n!(1\cdot2\cdot3\cdot\dots\cdot=n!)$. Можно ли вычеркнуть из произведения $1!\cdot2!\cdot3!\cdot\dotsb\cdot100!$ один из факториалов так, чтобы произведение оставшихся было квадратом целого числа?

Задача 16. Про натуральные числа m и n известно, что $3n^3= 5m^2$. Найдите наименьшее возможное значение $m+n$.

Задача 17. Докажите, что число $n^3-n$ делится на 6 при всех целых $n$.     

Задача 18. Докажите, что существует бесконечно много чисел, которые не  представимы в
виде суммы двух квадратов.

Задача 19. Докажите, что число, в десятичной записи которого участвуют три единицы и
несколько нулей, не может быть квадратом.

Задача 20. Докажите,   что   число  МЫШКА + КАМЫШ,   где   одинаковые   буквы   означают
одинаковые цифры, а разные буквы – разные цифры,  не кратно 101.

Задача 21. На доске написано 30 различных натуральных чисел, десятичная запись каждого из которых оканчивается или на цифру 2, или на цифру 6. Сумма написанных чисел равна 2454.
\begin{enumerate}
	\item Может ли на доске быть поровну чисел, оканчивающихся на 2 и на 6?
	\item Может ли ровно одно число на доске оканчиваться на 6?
	\item Какое наименьшее количество чисел, оканчивающихся на 6, может быть записано на доске?
\end{enumerate}
	
\end{document}